% Real-Time Scheduling

In mixed-criticality mission systems,
execution threads often 
have strict confidentiality, integrity, and availability requirements. 
Under certain conditions,
timing channels can be manipulated to violate these security requirements. 
For example, temporal interference between timing channels
can reduce availability of time-critical functionality 
and weaken the integrity of controls by inducing selective jitter.
As a result, a compromised component could dominate the processor, 
preventing other components from completing their tasks, placing the security 
of the entire platform in jeopardy. 
Simple scheduling approaches that attempt to use priority schemes to mitigate 
this impact only protect the highest priority threads and leave the other
threads vulnerable.

\emph{Temporal isolation} is a more secure technique to restrict timing channels and
reduce temporal interference between software threads executing on the same platform. 
\briefcase\ achieves temporal isolation by leveraging
prior work from safety and security-critical disciplines, such as
avionics, where temporal isolation in real-time scheduling has been
deployed for decades.
We implemented a static cyclic scheduling
approach using the \selFour\ domain scheduler, where a fixed schedule
defines an ordered sequence of static execution slots. Each slot has a
duration and a partition identifier.  \selFour\ ensures that the
temporal domain, and any threads running in it, will not exceed their defined
time allotments. Thus, each thread executes within its own
well-defined timing constraints without interference from other threads.
System designers are responsible for the
creation of a valid static cyclic schedule that
satisfies all application timing requirements.
To guarantee that all real-time deadlines are met, system designers must 
also ensure (by test or analysis) that thread worst-case execution times
do not exceed their scheduled allotments.  This is not currently part of 
the \briefcase\ tools.  

\briefcase\ generates a start-of-frame synchronization signal for each thread using
a special thread called the Pacer, which sends periodic signals
to each thread. Each application thread blocks until it receives its
signal from the Pacer. The application thread runs to completion, and then
blocks again on the Pacer signal for the next iteration. 
Each thread subsequently executes
exactly once during its statically scheduled time slice.

The new \selFour\ Mixed-Criticality Systems (MCS) variant provides
additional capability that can support temporally isolated
real-time systems. 
As part of \briefcase\ we developed a proof-of-concept static cyclic
scheduler for MCS.
It includes a start-of-frame signal, which
eliminates the need for the Pacer component. It also includes kernel
level support for flexible dynamic scheduling that satisfies some
real-time properties, such as period and execution time.

