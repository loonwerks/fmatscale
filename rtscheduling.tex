% Real-Time Scheduling

\emph{Temporal isolation} is a technique to reduce timing channels and temporal 
interference between software threads executing on the same platform. 
In certain conditions,
timing channels can be used to violate confidentiality requirements. 
Temporal interference can reduce availability of time-critical functionality, 
and impact integrity of controls by inducing selective jitter. 
For example, if one component is compromised, it could dominate the processor, 
preventing other components from completing their tasks, placing the security 
of the entire platform in jeopardy. 
In mixed-criticality mission systems, 
multiple threads necessary for mission success often 
have strict confidentiality, integrity, and availability requirements. 
Simple scheduling approaches that attempt to use priority schemes to mitigate 
this impact only protect the highest priority threads.

\briefcase\ achieves temporal isolation in a real-time
environment in which threads execute within
their own scheduled timing constraints without interference from other
threads hosted on the same platform. To accomplish this, we leveraged
prior work from safety and security-critical disciplines, such as
avionics, where temporal isolation in real-time scheduling has been
deployed for decades. We implemented a static cyclic scheduling
approach using the \selFour\ domain scheduler, where a fixed schedule
defines an ordered sequence of static execution slots. Each slot has a
duration and a partition identifier.  \selFour\ ensures that the
temporal domain, and any threads running in it, will not exceed their defined
time allotments. Creation of a valid static cyclic schedule that
satisfies all the application's timing requirements is the
responsibility of the system designers.

\briefcase\ generates a start-of-frame synchronization signal for each thread using
a special thread called the Pacer, which sends periodic signals
to each thread. Each application thread blocks until it receives its
signal from the Pacer. The thread runs to complete, and then
blocks again on the Pacer signal for the next iteration. 
Each thread subsequently executes
exactly once during its statically scheduled time slice.

The new \selFour\ Mixed-Criticality Systems (MCS) variant provides
additional capability that can support temporally isolated
real-time systems. 
As part of \briefcase\ we developed a proof-of-concept static cyclic
scheduler for MCS.
It includes a start-of-frame signal, which
eliminates the need for the Pacer component. It also includes kernel
level support for flexible dynamic scheduling that satisfies some
real-time properties, such as period and execution time.

