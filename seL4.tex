% Corey, Gerwin -  Budget: 1 column
% word budget: ~ 400 words

The seL4 microkernel~\cite{sel4-sosp09} is a lightweight, fast, and secure
operating system (OS) kernel. Its implementation is fully formally verified;
from high-level security properties down to the binary level. It was the first
OS kernel with this degree of formal verification, and after more than a decade
of further research and engineering is still not only the leading formally
verified OS kernel, but also the fastest OS kernel on the Arm architecture.

Its formal verification makes seL4 the ideal basis for high-assurance systems.
It is in itself a demonstration of the level of fidelity and scale formal
verification can achieve~\cite{sel4-tocs14}. It supports multiple architectures
(Arm, x86-64, RISC-V), deep security properties such as integrity,
confidentiality and availability, and comes with formally verified user-level
system initialization. As one of its multiple available OS personalities, it
also offers the user-level CAmkES component system that provably achieves isolation
between statically specified components.

The formal proofs about seL4 and the corresponding user-level components measure
over one million lines of proof script in total. They constitute one of the
largest continually maintained formal proof artifacts in existence and provide a
rich target for new techniques in proof engineering, proof repair, and
automation for constructing new proofs about software as well as maintaining
existing large-scale proof artifacts.

While it is essential to build a high-assurance system on a high-assurance OS
kernel, this is not the main feature seL4 provides for systems engineering --- a
simpler real-time OS might be formally verified, but would not be sufficient for the
engineering method described in this paper. The true power of seL4 lies in its
ability to scale formal analysis and verification to the much larger code bases
that make up entire systems. It does so by providing strong isolation between
user-level components~\cite{sel4-cacm18}.

This isolation means that components can be analyzed separately from each other
and be composed safely --- in this way seL4 provides the foundation that the soundness of the highly
automated analysis tools such as AGREE depend on. It makes it possible to run
entire untrusted virtual machines and securely monitor their behavior on the
same hardware. It makes it possible to provide filter and monitor components and
prove that these components cannot be tampered with by the components they protect.
And it makes it possible to guarantee that the limited communication channels that the analysis tools
assume to exist are the \emph{only} communication channels that are available to the
components in the system. The combination of these enables automated
high-level analysis with high assurance.
