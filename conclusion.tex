We have produced a Model-Based Systems Engineering (MBSE) environment
called BriefCASE, based on the Architecture Analysis and Design
Language (AADL). BriefCASE extends the Open Source AADL
Tool Environment (OSATE) to add new design, analysis, and code
generation capabilities based on formal methods and targeted at
high-assurance cyber-resilient system design. BriefCASE provides
access to two cyber requirements analysis tools (GearCASE
and DCRYPPS) that can examine AADL models to detect potential
cyber vulnerabilities and suggest requirements for mitigation.
A library of architectural transforms guides systems engineers
through automated model transformations that modify the
architecture to address these requirements, oftentimes inserting new
high-assurance components synthesized from formal
specifications using the SPLAT tool. Formal verification that the
transformed system model satisfies its cyber requirements is accomplished
via model checking using the Assume Guarantee Reasoning
Environment (AGREE). Cyber-resilient code implementing the
verified model is automatically generated using the HAMR
tool. If desired, this code can be targeted to the formally
verified seL4 secure microkernel.  Another novel aspect of our approach is
the use of an assurance argument embedded in the architecture model
itself to capture and document the design decisions made during this
process, along with associated rationale.

We have successfully demonstrated our BriefCASE methodology and tools on the
CH-47F helicopter Common Avionics Architecture System (CAAS), utilizing a team of CAAS
development engineers who had no previous experience with formal methods.  We
are currently incorporating lessons learned from this experience to improve the BriefCASE
methodology and tools, focusing on scalability and usability.


%open-source, prototype, extensible to other domains
%link to loonwerks.com
%significance -- recap innovations, scale, accessibility
%application to real systems