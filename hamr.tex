%% Text for the Infrastructure Code Generation subsection
% - assigned to John Hatcliff

The High Assurance Modeling and
Rapid engineering for embedded systems tool (\hamr)
\cite{hamr} is a multi-platform, multi-language
AADL code generation framework.  In the CASE project, HAMR is primarily used
to generate code for deployment for the seL4 microkernel, but system and component
prototyping is also supported utilizing HAMR's code generation capability
for the Java Virtual Machine (JVM) and Linux.  

Using seL4 as a foundation, HAMR enables
AADL to be used as a model-based development and systems
engineering framework for seL4-based applications.
One of the primary objectives is to support system builds that
leverage seL4 separation and information flow
guarantees to achieve the AADL-specified component isolation and
inter-component communication needed for cyber-resiliency.
HAMR ensures that seL4 is configured to permit the exact
inter-component information flows analyzed and visualized by Awas at
the model level.

For each AADL thread component, HAMR generates a thread code
skeleton and application programming interfaces (APIs) for communicating over the ports declared on
the component.  For components that are implemented manually, the
developer fills out the thread skeleton with application code,
calling the port APIs, libraries, and component-specific
developer-implemented support code as needed.
HAMR supports coding component application logic in either C,
Slang~\cite{slang} (a high-assurance subset of Scala that can be translated to
C), or CakeML. 
\remove{
Slang-implemented components can interface with
Scala or Java libraries and execute on the JVM.  Pure Slang
components can be translated to C and then deployed on HAMR-built
Linux or seL4 systems.  On CASE, Slang has been used for prototyping
components, but primary development of directly implemented
components has used C.}
For high-assurance components specified
using SPLAT, HAMR automatically integrates code generated from
SPLAT into the component binary code using CakeML's Foreign
Function Interface (FFI) mechanism.

HAMR generates component infrastructure and integration code
implementing the semantics of AADL-compliant thread scheduling,
thread dispatching, and port-based communication.
For port communication, shared memory communication (AADL
data ports), buffered messaging (AADL event data ports), and
buffered notification (AADL event ports) are supported.
HAMR code generation is staged using a translation architecture
that facilitates adding new backends for different target
platforms.   Almost all the infrastructure code is implemented
in Slang, which can then be used for JVM deployments or
translated to C for Linux or seL4 deployments.
The Slang-based implementation of the AADL run-time framework
can be viewed as a high-level reference implementation of AADL
semantics.   Automatic translation of this reference
implementation to C on different platforms helps establish
semantic consistency across those platforms. 
The AGREE contract language and SPLAT
code generation framework have been designed to align with the AADL
semantics reflected in this reference implementation.  

To generate deployments on seL4, HAMR makes use of the
component architecture for microkernel-based embedded systems
(CAmkES) code-generation framework.   The CAmkES input
language supports specification
of seL4 partitioning and communication topology using
component-oriented idioms.  The CAmKES translator generates an
seL4 \emph{Capability Description Language (CapDL)} file that
configures the seL4 kernel to support protected memory blocks
and permission-based reading and writing of each block as
indicated by CAmkES components and connections.
To realize the memory separation specified by the AADL
architecture description, HAMR generates CAmkES specifications
that: (a) reflect the AADL model's component and communication
topology, and (b) include additional components and communication
to support the AADL run-time infrastructure, in particular,
thread scheduling.
To enforce time partitioning, HAMR uses the seL4 domain
scheduler to support static cyclic scheduling. 

HAMR also provides automated support for configuring Linux-based
virtual machines as components within the seL4 deployed system. 
In the example UAV system, this feature has been used to sandbox untrusted legacy code
for the Mission Planner component.

%In CASE workflows,
%the FASTAR AADL scheduling tool from Adventium Labs was used
%to generate a candidate static schedule, and then the schedule
%was adjusted as needed based on timing experiments (this was
%needed primarily for VM components).

HAMR-generated seL4 systems can be executed on the QEMU emulator
before deployment to a development board or
production hardware.  This prototyping capability significantly speeds up development
iterations and enables the development of a sophisticated
automated regression testing framework.

While we have not yet completed full formal verification of
the HAMR Slang-based reference implementation and code generation
pipeline, information flow preservation is a key property that 
illustrates how one might approach formal verification of 
other important semantic properties.
Specifically, HAMR generates flow graphs reflecting the
inter-component information flow at both the AADL architecture
level and the seL4 CAmkES level.  In addition to multiple
visualizations and auditable traceability artifacts, HAMR
generates SMT-based representations of the flow graphs and
traceability relationships between them.  Formalized
theorems for information flow preservation can then be represented as
SMT assertions.  This enables SMT solvers to automatically prove
the following properties for any HAMR-generated seL4 deployment:

\begin{enumerate}
\item All AADL flows are implemented -- For every inter-component information
flow present in the AADL model, that flow is provisioned in the
generated seL4 kernel configuration.
\item No extraneous flows are implemented -- For
each inter-component flows provisioned in the seL4 kernel, that
flow is represented as an inter-component connection in the
AADL model.   
\end{enumerate}
These properties were chosen to focus on
cyber-resiliency needs, but other
key semantic properties of the AADL run-time can be incrementally
formally verified.
\remove{ e.g., by applying the Slang Logika formal
verification capability to the Slang reference implementation of the
AADL run-time.}  

\remove{
HAMR plays a key role in integrating formal methods
at different levels of abstraction at multiple points throughout
the development process, enabling those methods to be applied
at-scale in the system development process.   Using the
formally verified seL4 microkernel as a foundation, HAMR enables
AADL to be used as a model-based development and systems
engineering framework for seL4-based applications.
HAMR provides a semantically-consistent multi-platform code
generation process that enables: (a) formally verified components
(e.g., generated from SPLAT) to be correctly integrated and
deployed; and (b) formal specification and verification frameworks
like AGREE to be used to reason about both component and system
level properties.   The HAMR code generation architecture is
designed to support strong traceability and verification, as
illustrated by the information flow correspondence properties.  
The compositional and staged nature
of HAMR-based development supports scaling of formal approaches
by enabling them to be included in a component-wise manner and at
different levels of abstraction, while
also integrating parts of the implementation that are assured
using traditional methods.   In addition, the strong partitioning
of seL4 enables the controlled integration of untrusted components.
}

%SOME OF THIS LAST PARAGRAPH MIGHT BE BETTER IN THE CONCLUSION.


