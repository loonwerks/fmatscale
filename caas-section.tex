Collins Aerospace has successfully demonstrated our BriefCASE methodology and tools to
develop proof-of-concept enhancements for the  CH-47F helicopter Common Avionics
Architecture System (CAAS), utilizing a team of CAAS development engineers who had no
previous experience with formal methods.  As an integrated cockpit avionics suite, CAAS serves
as a prime example of modern air platform complexity with common avionics across a variety of
defense platforms including the Army's Mission Enhanced Little Bird (MELB)  and MH-60, the
Navy CH-53K, and the Air Force KC-135. As the primary designer and manufacturer of CAAS,
Collins Aerospace maintains complete design artifacts including boot loaders, operating systems,
software applications, hardware, and bus message specifications. CAAS offered an opportunity to
apply CASE tools across a variety of mission systems, including legacy components, flight critical
software, as well as new and evolving systems.

Our primary goal was to enable CAAS product engineers to employ the CASE tools to create an
operational mission scenario exhibiting cyber threats and mitigations that could be exercied using
the facilities of the Collins CH-47F System Integration Laboratory. The CH-47F demonstration system
for CASE focused on three areas: integrating pilot and soldier wireless tablet computers for
increased situational awareness, detecting attempted Automatic Dependent Surveillance (ADS-B)
traffic spoofing, as well as porting the high-assurance seL4 operating sytem to an existing Collins
CAAS product processing module. This processing module, or VPM, is provisioned to allow only vetted
traffic to/from the aforementioned wireless clients in order to provide situational awareness data
to pilots and soldiers, as well as to provide a platform for the execution of CASE-tool-generated
filters, monitors, attestation gates, etc.

The CAAS developers first developed an AADL model of the CH-47F CAAS system. They then added the
enhanced capabilities described above, resulting in a baseline architecture. The CAAS development
engineers then analyzed their baseline AADL architecture utilizing the GearCASE and DCRYPPS tools,
resulting in a set of cyber requirements. They then employed the BriefCASE AADL plug-in to provision
filters, monitors, attestation gates, seL4 isolates, etc., in order to satisfy the cyber
requirements. The specifications for the filters and monitors were developed by the CAAS developers
using the AGREE contract language, with assistance from the CASE developers, and the SPLAT tool was
invoked by BriefCASE in order to synthesize the monitors and filters from the AGREE specifications
with high assurance, as described previously. The HAMR tool was then invoked by BriefCASE in order
to synthesize the overall system, including the seL4 CAmkES components, CAmkES I/O, the processing
schedule, etc.

The tablet operating environment utilized either an off-the-shelf Android environment, or a Linux
guest OS running on seL4. This provided the CASE remote attestation team with differing challenges
when it came to the measurement and attestation of the tablet platform types, and provided the CASE
independent evaluators a variety of options for formulating attacks against the system. The ADS-B
anti-spoofing monitor developed by the CAAS team detected position/velocity trend inconsistencies,
duplicate traffic identifiers, as well as ``teleporting'' aircraft.

The port of the seL4 microkernel to the Collins VPM product was complicated by the need for network
proxy/adapters to bridge the untrusted wireless network to the rest of the sytem in a trustworthy
manner, including the ability to filter encrypted network traffic payloads. Fortunately, dedicated
processing cores were available on the VPM for the provisioning of these low side and high side
proxy/adapters. This code had to be manually generated due to the complexity of dealing with
off-the-shelf networking technologies; automating the synthesis of network adapter/proxy components
is future work.

The CAAS developers experienced tool immaturity/expressiveness/documentation issues, as was to be
expected for the first use of a research tool suite by product area developers. Initial estimates of
processing time required to run the monitoring components turned out to be overly optimistic,
requiring an optimization effort. Lack of detailed seL4 support for the target hardware led to more
extensive porting efforts for the VPM and tablets than was originally anticipated, and tablet
hardware instabilities further hampered the porting effort. But all in all, the CASE tools were able
to be productively used by Collins product area engineers to produce the CH-47F CAAS demonstration
system on time and within budget, providing the CASE developers with valuable feedback on the
strengths and weaknesses of the current CASE tool envionment.
