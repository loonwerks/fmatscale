%An important aspect of our work has been to structure formalizations and proofs by following
%the AADL model of the system. 
%In other work, we did this through the use of formal
%assume-guarantee contracts that correspond to the requirements for each component~\cite{HACMS}. 
Assurance activities for high-integrity systems in the aerospace domain are currently driven 
by industry and government standards such as DO-178C and MIL-HDBK-516C.  The use of assurance 
cases to show conformance to standards (or provide alternate means of assurance) is being pursued 
in separate research programs.  
In addition, we have found that in assuring the cyber-resiliency properties of aircraft designs we need to integrate
different kinds of evidence with varying levels of formality. This has been our motivation
% to explore assurance case methods.
for incorporating assurance case methods in BriefCASE.

In previous work, we developed the {\em Resolute} language and tool
%~\cite{resolute2014},
~\cite{resolute-destion} as a way to help engineers create an assurance
argument describing the steps taken during the design process to make the system safe and secure.
The Resolute syntax supports construction of assurance cases that comply with the Goal Structuring
Notation (GSN) v2 standard.
%~\cite{GSNv2} 
Claims are expressed as \textit{goals} and
\textit{strategies}, and can contain attributes such as \textit{context}, \textit{assumptions}, and
\textit{justification}. Claims can be marked \textit{undeveloped}, which Resolute interprets as an
unsupported claim, or with a \textit{solution}, which is an explicit assertion that the claim is
supported. Rather than being a separate document, a Resolute assurance case is part of the
architecture model and can refer to elements within the model. Since it is not a static
representation, it can ensure that the assurance argument remains consistent with the evolving
design.  

A partial assurance case for the hardened UAV surveillance system generated by Resolute 
is shown in \figref{fig:assurance-case}.  
The assurance case includes subtrees corresponding to the high-assurance components 
used to satisfy cyber-resiliency requirements and a subtree associated
% with HAMR code generation and traceability to seL4 separation guarantees, but it is collapsed in
% this view due to size constraints.  
with HAMR code generation and traceability to seL4 separation guarantees.  

\begin{figure*} 
\centering 
\includegraphics[width=\textwidth]{figs/assurance-case.jpg}
\caption{Resolute assurance case for hardened UAV surveilance system (partial). 
}
\label{fig:assurance-case} 
\end{figure*}

BriefCASE includes a library of Resolute assurance strategies, or \emph{patterns}, that align with
the CASE workflow. The patterns are instantiated with context from the AADL model and specify the
evidence required to support the cyber-resiliency goals of the system. For example, the
\texttt{add\_filter} strategy is automatically inserted into the assurance case when the
\textit{Filter} transformation is performed, and includes logical rules that Resolute uses to
determine whether the claim of well-formed messages is supported by evidence. The \texttt{add\_filter}
definition includes the following sub-goals (shown in \figref{fig:assurance-case}):
\begin{itemize} 
\item Component property implemented -- The filter has been implemented correctly 
to meet its AGREE specification, as shown by the proof produced by SPLAT.
\item Filter is connected -- The filter component is still present at the correct location in the model and has not been 
altered or deleted by subsequent design changes. 
\item Filter cannot be bypassed -- There is no alternate information flow in the model that 
would allow the filter to be bypassed and therefore not perform its function.   
\item AGREE properties are valid -- The filter specification has been verified by AGREE to meet its
intended purpose in the system.   
\end{itemize}

\remove{These sub-goals can be seen (in blue) in the zoomed in subtree in \figref{fig:assurance-case}.
The first two sub-goals are supported by evidence obtained by examining the structure of the model. 
If at a later time during development the model is inadvertently altered in a way that renders the transformation
ineffective, Resolute will be unable to substantiate the evidential statements and will
produce a failing assurance case.
%
%The third subgoal is satisfied through the use of SPLAT. SPLAT not only generates the implementation code for high-assurance components, but it also produces a proof that the generated code correctly implements its AGREE specification. 
For the third subgoal, Resolute uses the existence of the
SPLAT proof corresponding to the filter's AGREE specification
as evidence that the component was implemented correctly.}



% Resolute can determine whether an assurance case passes or fails


% Advocate?

% Show generated assurance case (in Advocate?)
