%Innovations
As part of the Cyber Assured Systems Engineering (CASE) project,
our team has developed a MBSE tool environment that integrates design, verification,
and code generation activities, enabling systems engineers to design-in cyber-resiliency
for complex cyber-physical systems. The {\em BriefCASE} tools capture our 
vision for how formal methods can be applied throughout the design and build process 
to create high-assurance cyber-resilient systems.  

A fundamental aspect of our approach is the use of 
architecture models to provide a framework for analyzing
system behavior and organizing the assurance evidence produced.  
AADL allows engineers to describe the important elements of distributed,
real-time, embedded systems (processors, memory, buses, processes, 
threads, and data interconnections) with sufficiently rigorous semantics
that can support formal reasoning.  

Proofs about models %are meaningless unless there is 
have less value without
some way to ensure
that the implementation retains the properties of the model.  The seL4 
microkernel \cite{sel4-sosp09}, used in both HACMS and CASE, is
formally verified from its high-level security properties down to its binary implementation.  
By targeting seL4 we ensure that system components cannot interact in unintended 
ways and the \textit{data flows} in the architecture model are enforced in the
final product.  

The main innovations of the BriefCASE tools and methodology are:

\begin{enumerate}

\item We provide automated architectural design patterns to address cyber-resiliency requirements, 
including synthesis of high-assurance components from formal specifications.

\item Our MBSE environment can target different operating systems including the seL4 microkernel, 
making its formal security guarantees easily accessible to developers. This ensures that 
the implementation produced is faithful to the modeled system. 

\item Our approach is based upon co-evolution of system design and assurance artifacts, so that
design changes automatically update the associated certification evidence.
\remove{
An {\em assurance case}
is a structured argument, supported by evidence, intended to justify that a system is acceptably assured 
relative to a concern (such as safety or security) in the intended operating environment.
An assurance case is embedded in the architecture model to capture and document the design decisions 
along with associated rationale.
}

\item Formal methods are integrated throughout the workflow, including requirements capture,
component synthesis, verification, code generation, and the seL4 microkernel itself. 

\end{enumerate}




%We have produced a prototype
%Model-Based Systems Engineering (MBSE) environment called
%BriefCASE which is based on the Architecture Analysis and Design
%Language (AADL). BriefCASE extends the Open Source AADL
%Tool Environment (OSATE) to add new design, analysis, and code
%generation capabilities targeted at building cyber-resilient systems.
%BriefCASE provides access to two analysis tools (GearCASE
%and DCRYPPS) that can examine AADL models to detect potential
%cyber vulnerabilities and suggest requirements for mitigation.
%A library of architectural transforms guides systems engineers
%through automated model transformations that modify the
%architecture to address these requirements, possibly inserting new
%high-assurance components into the system. Implementations for
%these new high-assurance components are synthesized from formal
%specifications using the Semantic Properties for Language and
%Automata Theory (SPLAT) tool. Formal verification that the
%transformed system model satisfies its cyber requirements is accomplished
%via model checking using the Assume Guarantee Reasoning
%Environment (AGREE). Cyber-resilient code implementing the
%verified model is automatically generated using the High Assurance
%Modeling and Rapid Engineering for Embedded Systems (HAMR)
%toolkit. If desired, this code can be targeted to the formally
%verified seL4 secure microkernel.
%A novel aspect of our approach is the use of an assurance argument
%embedded in the architecture model itself to capture and
%document the design decisions made during this process, along
%with associated rationale.
