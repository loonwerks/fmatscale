%Innovations

MBSE environment for high-assurance systems, providing access to FM tools at every level, integrating assurance evidence for co-evolving design and associated evidence.

\begin{enumerate}
\item Semi-automated architectural design patterns to address cyber-resiliency requirements, including synthesis of high-assurance components
\item MBSE environment that leverages seL4 security guarantees and makes this accessible
\item Co-evolution of system design and certification evidence/artifacts, organized by assurance argument
\item Integration of formal methods throughout the workflow (or is this part of one of the other innovations?)
\end{enumerate}

As part of the DARPA Cyber Assured Systems Engineering (CASE) program,
our team has developed design, analysis, and verification
tools that enable systems engineers to design-in cyber-resiliency
for complex cyber-physical systems. We have produced a prototype
Model-Based Systems Engineering (MBSE) environment called
BriefCASE which is based on the Architecture Analysis and Design
Language (AADL). BriefCASE extends the Open Source AADL
Tool Environmnet (OSATE) to add new design, analysis, and code
generation capabilities targeted at building cyber-resilient systems.
BriefCASE provides access to two analysis tools (GearCASE 
and DCRYPPS) that can examine AADL models to detect potential
cyber vulnerabilities and suggest requirements for mitigation.
A library of architectural transforms guides systems engineers
through automated model transformations that modify the
architecture to address these requirements, possibly inserting new
high-assurance components into the system. Implementations for
these new high-assurance components are synthesized from formal
specifications using the Semantic Properties for Language and
Automata Theory (SPLAT) tool. Formal verification that the
transformed system model satisfies its cyber requirements is accomplished
via model checking using the Assume Guarantee Reasoning
Environment (AGREE). Cyber-resilient code implementing the
verified model is automatically generated using the High Assurance
Modeling and Rapid Engineering for Embedded Systems (HAMR)
toolkit. If desired, this code can be targeted to the formally
verified seL4 secure microkernel.
A novel aspect of our approach is the use of an assurance argument
embedded in the architecture model itself to capture and
document the design decisions made during this process, along
with associated rationale.
