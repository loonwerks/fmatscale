%Motivation
\begin{figure*}
	\begin{center}
	  \includegraphics[width=\textwidth]{./figs/sw-hardened.png}
  \end{center}
	\caption{Cyber-resilient software architecture for UAV surveillance system 
	with new high-asurance components (green) and virtual machine hosting legacy software (red).} 
	\label{fig:sw-hardened} 
\end{figure*}

The example in \figref{fig:sw-hardened} shows an AADL model
% is a model-based design in the AADL OSATE tool 
of an unmanned air vehicle (UAV) for surveillance that was built with our \briefcase \ tools.  
We will use this example to explain how the tools work together to implement the system
and ensure its cyber-resiliency.  

The system includes a ground station computer and the aircraft, consisting of a mission computer and 
a flight control computer.  
The baseline (unhardened) mission computer included only four of the software components shown in \figref{fig:sw-hardened}: 
the radio for communication with the ground station (\emph{Radio}), 
the mission planning service (\emph{Planner}, provided as legacy software running on Linux), 
flight plan waypoint segmentation (\emph{WaypointManager}), 
and a serial interface to communicate with the flight control computer (\emph{UART}).

%\briefcase\ integrates four key formal methods in the model-based engineering workflow:
%verification via the Assume Guarantee REasoning Environment (\agree);
%synthesis of high-assurance components by means of the Semantic
%Properties of Language and Automata Theory tool (\splat); synthesis
%of inter-component communication by the High Assurance Modeling and
%Rapid engineering for embedded systems tool (\hamr); and the \selFour\ verified microkernel.
%Each component in the unhardened system, and the interface for the top-level software component, is
%formally specified with an AGREE contract stating assumptions on inputs and guarantees on outputs
%under the assumptions.
%\agree\ proves the unhardened implementation obeys the composition of these contracts. 
%
Cyber-threat analysis tools are used to analyze the unhardened functional model of the system, 
identifying the ground station and the mission planning
service as primary sources of cyber attacks.
%Seven new cyber requirements are added to the unhardened system that require ground station trust assessment, message integrity checks, and run-time monitoring of several conditions.
Seven new cyber requirements are introduced that address ground station trust, message integrity, and run-time behavior vulnerabilities.
The existing behavioral contracts in the unhardened system are strengthened to reflect these new requirements.
%\agree\ proves the unhardened system under these contracts fails verification. 

Design engineers use \briefcase\ to transform the baseline system model to that shown in
\figref{fig:sw-hardened}.  Automated model transformations address the cyber requirements by inserting  
new high-assurance components (shown in green) and targeting the seL4 kernel to enforce separation between components.
The \emph{AttestationManager} establishes the trustworthiness of ground stations while
the \emph{AttestationGate} only passes messages from trusted sources.  The three 
filters  (\emph{OR\_Filter}, \emph{LST\_Filter}, and \emph{AReq\_Filter}) 
only pass well-formed messages received from the Radio. Another filter (\emph{AResp\_Filter}) on the output of the Planner ensures 
that only well-formed flight plans are sent to the WaypointManager.  
Two run-time monitors alert the system to suspicious behaviors from the Planner such as flight plans that enter \textit{keep-out} zones or leave
\textit{keep-in} zones (\emph{Geofence\_Monitor}), or unresponsiveness (\emph{Response\_Monitor}).
The interface behavior of these high-assurance components, with the exception of the AttestationManager, is
specified with assume-guarantee contracts (e.g., a filter makes no assumptions on input and only
passes inputs that are syntactically well-formed).
%\agree\ proves the hardened system ensures the added cyber requirements.

The model is further transformed to move the mission planning service into a 
Linux virtual machine hosted on the \selFour \ microkernel. This permits us to run the
legacy mission planner code without modification or porting and isolate any 
unintended behaviors.  
The target platform requires a static real-time schedule that is provided in the model.
A transformation on the assume-guarantee contracts incorporates that schedule into the model
for verification of the cyber requirements.

Finally, cyber-resilient mission computer software is automatically generated from the verified 
AADL model for execution by a fully verified version of seL4 running on an ODROID-XU4 (ARM Cortex-A7 CPU).  
We successfully demonstrated the resilience of the system against a variety of cyber attacks. 

%\resolint\ certifies the hardened model is ready for synthesis.
%\splat\ synthesizes high-assurance components from their \agree\ contracts to the target platform.
%It includes proof certificates that the binaries have assumptions that are no stronger than those on
%the original contracts and guarantees that are no weaker than those on the original contracts (e.g.,
%safe substitution).
%\hamr\ synthesizes all the inter-component communication primitives from the AADL model.
%That synthesis includes a proof certificate that the resulting communication channels defined in
%\selFour\ are only those defined in the AADL model.
%
%\resolute\ builds an assurance case for the entire system.
%That assurance case includes evidences for every requirement including proof certificates from
%\agree, \splat, and \hamr.
