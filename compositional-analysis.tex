% Compositional Analysis

The Assume Guarantee Reasoning Environment (AGREE) is a compositional, assume-guarantee-style model
checker for AADL models \cite{compositional-analysis-agree}. AGREE attempts to prove properties about one layer
of the architecture using properties allocated to subcomponents. The composition is performed in
terms of assumptions and guarantees that are provided for each component. Assumptions describe the
expectations the component has on the environment, while guarantees describe bounds on the behavior
of the component. 

AGREE uses k-induction as the underlying algorithm for model checking.  AADL models and AGREE
contracts are first translated into the Lustre language, including verification conditions for consistency
and correctness of the contracts.  The model checker then attempts to find any model execution traces
that would violate these conditions using one of several Satisfiability Modulo Theories (SMT) solvers. 
If the model checker covers all reachable states in the model without finding a violation then the properties are proved. 

Once the system architecture has been modeled using AADL
and the component behavior is specified using \agree's assume-guarantee contracts,
we use the \agree{} model checker to verify the consistency of these contracts.

\begin{enumerate}
\item Component interfaces -- The output guarantees of each component must be strong enough to
satisfy the input assumptions of downstream components. 
\item Correctness of implementations -- The input assumptions of a system along with the 
output guarantees of its \emph{sub}-components must be strong enough to satisfy its output guarantees.
\end{enumerate}

For example, in \figref{fig:sw-hardened},
the input assumptions of the Waypoint Manager must be weaker than
the output guarantees of the Geofence monitor 
\textit{and} the output guarantees of the Mission Software must be inferrable from
its input assumptions combined with the output guarantees of its components.

This hierarchical strategy for reasoning about contracts,
or \emph{compositional analysis},
reduces the computational complexity of model checking
by breaking down the larger problem into more manageable fragments.
\remove{Verification of a system does not directly depend on the implementations of its components (or their sub-components),
but only on their contracts.  
In \figref{fig:sw-hardened}, each of the Mission Software components either contains subcomponents 
or has a corresponding source code implementation of its contract.  
Rather than reasoning monolithically about the interaction of all of these lower-level implementations, 
the compositional approach allows us to reason about each layer of the model separately.  
}
%More details on the use of composition to reduce verification complexity can be found in \cite{case-models-2021}.
