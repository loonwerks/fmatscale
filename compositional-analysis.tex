% Compositional Analysis

Once the system's architecture has been modeled using AADL
and component behavior specified using \agree's assume-guarantee contracts,
we use \agree's model checker to verify the consistency of these contracts.
The model checking process verifies that:
(a) the output guarantees of component implmentations are strong enough to
validate the input assumptions of downstream components;
(b) the input assumptions of a component along with the output guarantees of its \emph{sub}-components
are strong enough to validate its output guarantees.
For example, in \figref{fig:sw-hardened},
the input assumptions of the Waypoint Manager must be inferrable from
the output guarantees of the Geofence monitor, 
and the output guarantees of SW must be inferrable from
its input assumptions combined with the output guarantees of its subcomponents.

This hierachical strategy for reasoning about contracts,
i.e. \emph{Compositional Analysis},
reduces the computational complexity of model checking
by breaking down the larger problem into more manageable fragments (\cite{compositional-analysis-agree}).
In \figref{fig:sw-hardened}, SW's subcomponents can be broken into further subcomponents,
and this would, typically, multiply the complexity of the verification process.
But, in the case of Compositional Analysis (and \agree),
since verification of a component does not (directly) depend on the contracts of its \emph{sub}-subcomponents,
this multiplication of complexity does not occur.
More details can be found in \cite{case-models-2021}.
