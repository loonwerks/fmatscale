%Introduction
\chapterinitial{Aerospace systems engineers} are currently given few
development tools to understand and mitigate 
potential cybersecurity vulnerabilities.  Typically, they must rely on
process-oriented checklists and guidelines. Cyber vulnerabilities
are often discovered during penetration testing late in the
development process. Worse yet, they may be discovered
only after the product has been fielded, necessitating extremely
expensive and time-consuming remediation. This is not a
sustainable development model.

Fortunately, formal methods tools have advanced to the point that they can 
be used to address cybersecurity and cyber-resiliency design challenges
on real high-assuance systems at industrial scale, and do so much earlier in
the development cycle.  Our application domain is avionics and aerospace systems in general.  
This domain features large, real-time cyber-physical systems with the added 
complexities of performing safety-critical tasks as well as being exposed to 
a wide variety of cyber threats.  Furthermore, aerospace systems are subject 
to intense regulatory scrutiny due to the certification requirements of this domain. 

In previous work on the High-Assurance Cyber Military Systems (HACMS) project \cite{HACMS}
we demonstrated that formal methods could be used to dramatically improve the 
cyber-resiliency of real aircraft, including an unmanned military helicopter.  Our current
work is focused on automating the capabilities that we prototyped in the HACMS project
and extending the reach and scale of the formal methods design and verification approach.  

To this end, we have developed a model-based systems engineering (MBSE) 
environment that allows engineers to address a range of properties and 
manage system complexity through compositional analysis, integrating formal methods
at all levels of the design process.  MBSE processes utilize models as the primary vehicle for 
communication among the parties tasked with designing the system and as the primary 
design artifacts for requirements, verification, and code generation.  

Our tools are based on the 
Architecture Analysis and Design Language (AADL) and extend the Open Source
AADL Tool Environment (OSATE) \cite{OSATE}.  The tools are specifically designed 
to bridge the gap between a user-level modeling language accessible to systems 
engineers and the highly specialized, formally verified code that implements the operating system (OS)
kernel and other high-assurance components.   

By using these tools to build real avionics systems, we show 
that current formal methods tools are practical, effective, and scalable to significant 
high-assurance applications in the aerospace industry.  
