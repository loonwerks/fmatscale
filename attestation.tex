Semantic remote attestation is a technique for remotely establishing
trust in running software.  An appraiser requests an attestation from
a target, receives evidence in response it the request, and appraises
the evidence to determine
trust~\cite{Haldar:04:Semantic-Remote,Coker::Principles-of-R}. Because
remote attestation does not require modification of its measurement
target, StairCASE utilizes it to establish trust in legacy software
that cannot otherwise be verified.  We construct a verified remote
attestation infrastructure around the legacy target that generates
run-time and boot-time evidence.

The approach adds three attestation managers to the seL4-based
architecture used to harden the attestation target.  Each attestation
manager executes protocols specified by Copland~\cite{Ramsdell:2019aa}
phrases.  Copland is a formally specified language designed for
writing attestation protocols that are both verifiable and executable.
The attestation managers themselves are written and verified using
CakeML~\cite{Kumar:2014:CVI:2535838.2535841} and
Coq~\cite{Bertot:2013aa}.

The UserAM runs as a Linux process on the seL4 virtual machine.  It's
responsibilities are responding to attestation requests from off
platform, measuring the hardened application, and requesting
attestations from the platform.  When the UserAM receives an
attestation request it responds by executing an attestation protocol
that measures the hardened application, requests measurements from the
PlatformAM, and signs the result and a nonce from the request.  The
resulting evidence package is returned to the requesting appraiser.

Because the UserAM runs as a process in the Linux kernel, it cannot be
trusted \emph{a priori}.  A PlatformAM is introduced to perform an
attestation on the Linux kernel.  The PlatformAM runs as a
CAmkES component separated from the Linux kernel.  seL4's guaranteed
separation properties provide assurance that the PlatformAM cannot be
interfered with by other platform software.  The PlatformAM only
responds to requests from the UserAM and similarly runs a protocol
that produces signed results.

A third attestation manager is added to the appraiser to make
attestation requests and appraise results.  A protocol and nonce are
sent to the target, evidence is returned, and results appraised to
determine trust.  The attestation manager then communicates appraisal
results to the platform communicating with the remote target.  In this
way the platform appraises a remote target before trusting it to
behave as expected.

Using the remote attestation system to harden a platform involves
writing application specific attestation and appraisal components.
New measurers are constructed for the UserAM that measure the running
application along with corresponding appraisal code.  The PlatformAM
and attestation architecture remain the same across applications.  The
overhead required for hardening is thus minimized.