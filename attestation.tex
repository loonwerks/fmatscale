Semantic remote attestation is a technique for  establishing
trust in software running on a non-local computer.  
An appraiser requests an attestation from
a target, receives evidence in response to the request, and appraises
the evidence to determine
trust~\cite{Coker::Principles-of-R}. Because
remote attestation does not require modification of its measurement
target, we utilize it to establish trust in legacy software
that cannot otherwise be verified.  We construct a verified remote
attestation infrastructure around the legacy target that generates
run-time and boot-time evidence.

In the UAV example, we wish to ensure that the aircraft will only accept
commands from trustworthy (non-compromised) ground stations.  
Our approach adds three attestation managers to the seL4-based
architecture in \figref{fig:sw-hardened}.  Each attestation
manager executes protocols specified by Copland~\cite{Ramsdell:2019aa}
phrases.  Copland is a formal specification language designed for
writing attestation protocols that are both verifiable and executable.
The attestation managers themselves are written and verified using
CakeML and Coq.

The \emph{AttestationManager} on the Mission Computer makes
attestation requests and appraises the results.  A protocol request and nonce are
sent to the remote target (in this case, the Ground Station), 
evidence is returned, and results appraised to
determine trust.  If the appraisal is successful, the identifier for the 
target is added to a list of trusted computers whose messages 
will be allowed to pass through the \emph{AttestationGate}.  
%The AttestationManager then communicates appraisal
%results to the platform communicating with the remote target.  In this
%way the platform appraises a remote target before trusting it to
%behave as expected.

Two other attestation managers must be added on the Ground Station.  
The Ground Station software has been modified to run in a Linux virtual 
machined hosted on \selFour.  
The UserAM runs as a Linux process on this virtual machine.  Its
responsibilities include responding to attestation requests from the UAV, 
measuring the Ground Station application software, and requesting
attestations from the Ground Station platform (the Linux kernel).  When the UserAM receives an
attestation request it responds by executing an attestation protocol
that measures the application, requests measurements from the
PlatformAM, signs the result, and generates a nonce from the request.  The
resulting evidence package is returned to the requesting appraiser on the UAV.

Because the UserAM runs as a Linux process, it cannot be
trusted \emph{a priori}.  A PlatformAM is introduced to perform an
attestation of the Linux kernel (and the UserAM).  The PlatformAM runs as a
\selFour \ component outside of the Linux virtual machine 
separated from the Linux kernel.  \selFour's guaranteed
separation properties provide assurance that the PlatformAM cannot be
compromised by other platform software.  The PlatformAM only
responds to requests from the UserAM and similarly runs a protocol
that produces signed results.

Using the remote attestation system to harden a platform involves
writing application-specific attestation and appraisal components.
New measurers are constructed for the UserAM that measure the running
application along with corresponding appraisal code.  The PlatformAM
and attestation architecture remain the same across applications.
Thus, the overhead required for the implementation is minimized.