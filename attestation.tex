Semantic remote attestation is a technique for remotely establishing
trust in running software.  An appraiser requests an attestation from
a target, receives evidence in response it the request, and appraises
the evidence to determine
trust~\cite{Haldar:04:Semantic-Remote,Coker::Principles-of-R}. Because
remote attestation does not require modification of its measurement
target, StairCASE utilizes it to establish trust in legacy software
that cannot otherwise be verified.  We construct a verified remote
attestation infrastructure around the legacy target that generates
run-time and boot-time evidence.

The approach adds attestation managers to the seL4-based architecture
used to harden the attestation target.  Each attestation manager
executes protocols specified by Copland~\cite{Ramsdell:2019aa}
phrases.  Copland is a formally specified language designed for
writing attestation protocols that are both verifiable and executable.
The attestation managers themselves are written and verified using
CakeML~\cite{Kumar:2014:CVI:2535838.2535841} and
Coq~\cite{Bertot:2013aa}.

A separate attestation manager is added to the appraiser to make
attestation requests and appraise results.  A protocol and nonce are
sent to the target, evidence is returned, and results appraised to
determine trust.  The attestation manager then communicates appraisal
results to the platform communicating with the remote target.  In this
way the platform appraises a remote target before trusting it to
behave as expected.
